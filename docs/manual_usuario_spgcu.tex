\documentclass[12pt,a4paper]{article}
\usepackage[utf8]{inputenc}
\usepackage[spanish]{babel}
\usepackage{graphicx}
\usepackage{hyperref}
\usepackage{geometry}
\usepackage{fancyhdr}
\usepackage{xcolor}
\usepackage{listings}
\usepackage{tocloft}
\usepackage{enumitem}

\geometry{margin=2.5cm}

\definecolor{primaryblue}{HTML}{31436B}
\definecolor{lightgray}{HTML}{F3F4F6}

\hypersetup{
    colorlinks=true,
    linkcolor=primaryblue,
    urlcolor=primaryblue
}

\pagestyle{fancy}
\fancyhf{}
\fancyhead[L]{\textbf{Manual de Usuario - SPGCU}}
\fancyhead[R]{\thepage}
\fancyfoot[C]{Universidad Nacional de Piura}

\title{
    \vspace{-2cm}
    % \includegraphics[width=4cm]{logo_unp.png} \\ % Descomentar cuando tengas el logo
    \vspace{1cm}
    \vspace{1cm}
    \textbf{\Huge Manual de Usuario} \\
    \vspace{0.5cm}
    \textbf{\Large Sistema de Programación y Gestión del Comedor Universitario} \\
    \vspace{0.3cm}
    \textbf{(SPGCU)}
}
\author{
    \textbf{Oficina de Bienestar Universitario} \\
    Universidad Nacional de Piura
}
\date{Diciembre 2025}

\begin{document}

\maketitle
\thispagestyle{empty}
\newpage

\tableofcontents
\newpage

% ============================================================
\section{Introducción}
% ============================================================

El Sistema de Programación y Gestión del Comedor Universitario (SPGCU) es una plataforma web diseñada para automatizar y optimizar los procesos relacionados con el servicio de alimentación para estudiantes beneficiarios de la Universidad Nacional de Piura.

\subsection{Objetivos del Sistema}
\begin{itemize}
    \item Gestionar convocatorias para nuevos beneficiarios del comedor universitario.
    \item Registrar y evaluar postulaciones de estudiantes.
    \item Programar y controlar la asistencia al comedor mediante códigos QR.
    \item Administrar menús diarios (desayuno, almuerzo, cena).
    \item Generar reportes de asistencia y beneficiarios.
\end{itemize}

\subsection{Roles de Usuario}
El sistema contempla tres roles principales:

\begin{enumerate}
    \item \textbf{Estudiante}: Puede postular a convocatorias, ver su QR, programar comidas y solicitar justificaciones.
    \item \textbf{Administrativo (Bienestar Universitario)}: Gestiona convocatorias, evalúa entrevistas, administra menús y registra asistencias.
    \item \textbf{Administrador}: Acceso completo al sistema, incluyendo gestión de usuarios y configuraciones avanzadas.
\end{enumerate}

% ============================================================
\section{Acceso al Sistema}
% ============================================================

\subsection{Requisitos del Sistema}
\begin{itemize}
    \item Navegador web moderno (Chrome, Firefox, Edge).
    \item Conexión a Internet estable.
    \item Para escaneo QR: Dispositivo con cámara.
\end{itemize}

\subsection{Inicio de Sesión}
\begin{enumerate}
    \item Acceda a la URL del sistema: \texttt{http://spgcu.unp.edu.pe}
    \item Ingrese su \textbf{DNI} y \textbf{Contraseña}.
    \item Haga clic en \textbf{Iniciar Sesión}.
\end{enumerate}

\begin{figure}[h]
    \centering
    % \includegraphics[width=0.8\textwidth]{login.png}
    \caption{Pantalla de Inicio de Sesión}
    \label{fig:login}
\end{figure}

\subsection{Registro de Nuevo Usuario (Estudiantes)}
\begin{enumerate}
    \item En la pantalla de login, haga clic en \textbf{Registrarse}.
    \item Complete el formulario con sus datos personales:
    \begin{itemize}
        \item DNI (8 dígitos)
        \item Nombres y Apellidos
        \item Código de Estudiante
        \item Correo Electrónico Institucional
        \item Contraseña (mínimo 8 caracteres)
    \end{itemize}
    \item Haga clic en \textbf{Crear Cuenta}.
\end{enumerate}

\begin{figure}[h]
    \centering
    % \includegraphics[width=0.8\textwidth]{registro.png}
    \caption{Formulario de Registro de Estudiante}
    \label{fig:registro}
\end{figure}

% ============================================================
\section{Módulo de Estudiante}
% ============================================================

\subsection{Dashboard del Estudiante}
Al iniciar sesión, el estudiante verá un panel con accesos rápidos a:
\begin{itemize}
    \item Convocatorias Activas
    \item Mi Código QR
    \item Programar Comidas
    \item Mis Justificaciones
\end{itemize}

\begin{figure}[h]
    \centering
    % \includegraphics[width=0.9\textwidth]{dashboard_estudiante.png}
    \caption{Dashboard del Estudiante}
    \label{fig:dashboard_estudiante}
\end{figure}

\subsection{Postular a una Convocatoria}
\begin{enumerate}
    \item Desde el menú lateral, seleccione \textbf{Postulaciones} $\rightarrow$ \textbf{Nueva Postulación}.
    \item Seleccione la convocatoria activa.
    \item Complete el formulario socioeconómico:
    \begin{itemize}
        \item Ingreso Familiar Mensual (S/.)
        \item Número de Miembros del Hogar
        \item Condición de Vivienda (Propia, Alquilada, etc.)
    \end{itemize}
    \item Adjunte los documentos requeridos (PDF, máximo 5MB cada uno).
    \item Haga clic en \textbf{Enviar Postulación}.
\end{enumerate}

\begin{figure}[h]
    \centering
    % \includegraphics[width=0.9\textwidth]{formulario_postulacion.png}
    \caption{Formulario de Postulación}
    \label{fig:postulacion}
\end{figure}

\textbf{Nota:} Solo puede postular a una convocatoria activa a la vez.

\subsection{Mi Código QR}
Una vez aprobado como beneficiario, podrá acceder a su código QR:

\begin{enumerate}
    \item Vaya a \textbf{Mi QR} desde el menú o dashboard.
    \item Se mostrará su código QR personal junto con sus datos.
    \item Presente este código en el comedor para registrar su asistencia.
\end{enumerate}

\begin{figure}[h]
    \centering
    % \includegraphics[width=0.6\textwidth]{mi_qr.png}
    \caption{Código QR del Estudiante Beneficiario}
    \label{fig:mi_qr}
\end{figure}

\textbf{Importante:}
\begin{itemize}
    \item El código QR es personal e intransferible.
    \item Suba el brillo de su pantalla al máximo para facilitar el escaneo.
\end{itemize}

\subsection{Programar Comidas Semanales}
\begin{enumerate}
    \item Acceda a \textbf{Comedor} $\rightarrow$ \textbf{Mi Horario}.
    \item Visualice el calendario semanal con los menús disponibles.
    \item Haga clic en las comidas que desea reservar (máximo 3 por día).
    \item Haga clic en \textbf{Guardar Selección}.
\end{enumerate}

\begin{figure}[h]
    \centering
    % \includegraphics[width=0.9\textwidth]{programar_comidas.png}
    \caption{Calendario de Programación de Comidas}
    \label{fig:programar_comidas}
\end{figure}

\textbf{Restricciones:}
\begin{itemize}
    \item No puede reservar comidas para fechas pasadas.
    \item Si no asiste a una comida reservada, se contará como \textbf{Inasistencia}.
    \item Acumular 3 inasistencias resultará en \textbf{suspensión temporal}.
\end{itemize}

\subsection{Solicitar Justificación}
Si no puede asistir a una comida reservada, puede solicitar justificación:

\begin{enumerate}
    \item Vaya a \textbf{Justificaciones}.
    \item Haga clic en \textbf{Nueva Justificación}.
    \item Seleccione la fecha a justificar.
    \item Escriba el motivo detallado.
    \item Adjunte documentos de respaldo si los tiene.
    \item Envíe la solicitud.
\end{enumerate}

\begin{figure}[h]
    \centering
    % \includegraphics[width=0.9\textwidth]{solicitar_justificacion.png}
    \caption{Formulario de Solicitud de Justificación}
    \label{fig:justificacion}
\end{figure}

El estado de su justificación puede ser:
\begin{itemize}
    \item \textcolor{gray}{\textbf{Pendiente}}: En espera de revisión.
    \item \textcolor{green}{\textbf{Aprobada}}: La inasistencia ha sido justificada.
    \item \textcolor{red}{\textbf{Rechazada}}: No se aceptó la justificación.
\end{itemize}

% ============================================================
\section{Módulo de Administrador / Administrativo}
% ============================================================

\subsection{Dashboard Administrativo}
Los usuarios con rol Administrador o Administrativo verán un panel con estadísticas y accesos rápidos a:

\begin{itemize}
    \item Convocatorias
    \item Postulaciones Pendientes
    \item Entrevistas
    \item Lista de Beneficiarios
    \item Gestión de Menús
    \item Escáner de Asistencia
    \item Reportes
\end{itemize}

\begin{figure}[h]
    \centering
    % \includegraphics[width=0.9\textwidth]{dashboard_admin.png}
    \caption{Dashboard del Administrador}
    \label{fig:dashboard_admin}
\end{figure}

\subsection{Gestión de Convocatorias}
\begin{enumerate}
    \item Vaya a \textbf{Convocatorias}.
    \item Para crear una nueva, haga clic en \textbf{Nueva Convocatoria}.
    \item Complete los campos:
    \begin{itemize}
        \item Nombre de la Convocatoria
        \item Fecha de Inicio y Fin
        \item Cupos Disponibles
        \item Descripción y Requisitos
    \end{itemize}
    \item Guarde la convocatoria.
\end{enumerate}

\begin{figure}[h]
    \centering
    % \includegraphics[width=0.9\textwidth]{gestion_convocatorias.png}
    \caption{Lista de Convocatorias}
    \label{fig:convocatorias}
\end{figure}

\subsection{Evaluación de Postulaciones}
\begin{enumerate}
    \item Acceda a \textbf{Postulaciones}.
    \item Filtre por estado (Pendiente, Verificado, etc.).
    \item Haga clic en una postulación para ver el detalle.
    \item Revise los documentos adjuntos.
    \item Asigne un estado: \textbf{Verificado}, \textbf{Apto para Entrevista}, o \textbf{Rechazado}.
\end{enumerate}

\begin{figure}[h]
    \centering
    % \includegraphics[width=0.9\textwidth]{evaluar_postulacion.png}
    \caption{Detalle de Postulación para Evaluación}
    \label{fig:evaluar_postulacion}
\end{figure}

\subsection{Programación y Evaluación de Entrevistas}
\begin{enumerate}
    \item Vaya a \textbf{Entrevistas}.
    \item Para postulantes ``Apto para Entrevista'', programe fecha y hora.
    \item Después de la entrevista, registre el resultado:
    \begin{itemize}
        \item \textbf{Apto}: El postulante pasa a ser Beneficiario.
        \item \textbf{No Asistió}: Se marca como inasistente.
    \end{itemize}
\end{enumerate}

\begin{figure}[h]
    \centering
    % \includegraphics[width=0.9\textwidth]{entrevistas.png}
    \caption{Lista de Entrevistas Programadas}
    \label{fig:entrevistas}
\end{figure}

\subsection{Lista de Beneficiarios}
\begin{enumerate}
    \item Acceda a \textbf{Beneficiarios}.
    \item Visualice la lista de estudiantes con beneficio activo.
    \item Use la barra de búsqueda para filtrar por nombre, código o DNI.
    \item Puede \textbf{Exportar a Excel} la lista completa.
    \item La columna \textbf{Faltas} muestra el conteo de inasistencias:
    \begin{itemize}
        \item \textcolor{green}{\textbf{0 Faltas}}: Sin problemas.
        \item \textcolor{orange}{\textbf{1-2 Faltas}}: Advertencia.
        \item \textcolor{red}{\textbf{3+ Faltas}}: Suspensión automática.
    \end{itemize}
    \item Haga clic en el contador de faltas para ver el detalle y \textbf{Quitar Falta} manualmente si corresponde.
\end{enumerate}

\begin{figure}[h]
    \centering
    % \includegraphics[width=0.9\textwidth]{lista_beneficiarios.png}
    \caption{Lista de Beneficiarios con Conteo de Faltas}
    \label{fig:beneficiarios}
\end{figure}

\subsection{Gestión de Menús}
\begin{enumerate}
    \item Vaya a \textbf{Menús} $\rightarrow$ \textbf{Calendario}.
    \item Haga clic en un día futuro para agregar menús.
    \item Complete:
    \begin{itemize}
        \item Tipo: Desayuno, Almuerzo o Cena.
        \item Hora de Inicio y Fin.
        \item Descripción del menú (opcional).
    \end{itemize}
    \item Guarde el menú.
\end{enumerate}

\begin{figure}[h]
    \centering
    % \includegraphics[width=0.9\textwidth]{calendario_menus.png}
    \caption{Calendario de Administración de Menús}
    \label{fig:calendario_menus}
\end{figure}

\textbf{Nota:} No se pueden crear menús para fechas pasadas.

\subsection{Escáner de Asistencia (QR)}
\begin{enumerate}
    \item Acceda a \textbf{Escáner} desde el menú.
    \item Seleccione la cámara a utilizar (si tiene varias).
    \item Haga clic en \textbf{Iniciar Escáner}.
    \item Apunte la cámara al código QR del estudiante.
    \item El sistema validará automáticamente:
    \begin{itemize}
        \item Que el estudiante sea beneficiario activo.
        \item Que tenga reserva para la comida actual.
        \item Que esté dentro del horario permitido.
    \end{itemize}
    \item Si es válido, mostrará confirmación con el nombre del estudiante y el menú registrado.
\end{enumerate}

\begin{figure}[h]
    \centering
    % \includegraphics[width=0.7\textwidth]{escaner_qr.png}
    \caption{Escáner de Asistencia por QR}
    \label{fig:escaner}
\end{figure}

\subsection{Lista de Asistencia del Día}
\begin{enumerate}
    \item Vaya a \textbf{Asistencia} $\rightarrow$ \textbf{Lista de Hoy}.
    \item Visualice las reservas del día separadas por tipo de comida.
    \item Use el \textbf{interruptor (switch)} para marcar manualmente como ``Presente'' o ``Ausente''.
    \item Puede \textbf{Imprimir} o \textbf{Exportar a Excel} la lista.
\end{enumerate}

\begin{figure}[h]
    \centering
    % \includegraphics[width=0.9\textwidth]{asistencia_hoy.png}
    \caption{Lista de Asistencia del Día con Switch de Control}
    \label{fig:asistencia_hoy}
\end{figure}

\subsection{Gestión de Justificaciones}
\begin{enumerate}
    \item Acceda a \textbf{Justificaciones}.
    \item Visualice las solicitudes pendientes.
    \item Haga clic en una justificación para ver el detalle.
    \item Apruebe o Rechace la solicitud.
    \item Si se aprueba, la inasistencia se convierte en ``Justificado''.
\end{enumerate}

\begin{figure}[h]
    \centering
    % \includegraphics[width=0.9\textwidth]{gestionar_justificaciones.png}
    \caption{Gestión de Justificaciones}
    \label{fig:gestionar_justificaciones}
\end{figure}

% ============================================================
\section{Procesos Automáticos}
% ============================================================

\subsection{Marcado Automático de Inasistencias}
El sistema ejecuta un proceso automático que:

\begin{enumerate}
    \item Al finalizar el día, revisa las reservas ``Programadas'' que no fueron marcadas como ``Asistió''.
    \item \textbf{Regla}: Si el estudiante asistió al menos a 1 comida del día, las demás reservas se \textbf{perdonan}.
    \item Si no asistió a ninguna comida, las reservas se marcan como \textbf{Falta}.
    \item Si un estudiante acumula \textbf{3 o más faltas}, se \textbf{suspende automáticamente}.
\end{enumerate}

\subsection{Marcado en Tiempo Real}
Adicionalmente, el sistema puede ejecutar un proceso ``en tiempo real'' que:

\begin{enumerate}
    \item Busca menús cuya hora de fin ya pasó (HOY).
    \item Marca como ``Falta'' a quienes quedaron en estado ``Programado''.
\end{enumerate}

Este proceso puede ejecutarse cada 15 minutos para mantener la lista actualizada.

% ============================================================
\section{Preguntas Frecuentes}
% ============================================================

\subsection{¿Qué hago si olvidé mi contraseña?}
Contacte a la Oficina de Bienestar Universitario para solicitar el restablecimiento.

\subsection{¿Puedo cancelar una reserva de comida?}
Actualmente no es posible cancelar reservas. Si no puede asistir, solicite una justificación con anticipación.

\subsection{¿Qué pasa si me suspenden?}
Deberá acercarse a la Oficina de Bienestar Universitario para regularizar su situación.

\subsection{¿El código QR tiene fecha de vencimiento?}
El código QR es válido mientras su estado como beneficiario esté activo.

% ============================================================
\section{Soporte Técnico}
% ============================================================

Para cualquier problema técnico o consulta, contacte a:

\begin{itemize}
    \item \textbf{Oficina de Bienestar Universitario}
    \item Correo: bienestar@unp.edu.pe
    \item Teléfono: (073) 123456
    \item Horario de Atención: Lunes a Viernes, 8:00 AM - 4:00 PM
\end{itemize}

\vspace{2cm}
\hrule
\vspace{0.5cm}
\centering
\textit{Documento generado el \today}

\end{document}
