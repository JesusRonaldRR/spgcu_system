\documentclass[12pt,a4paper]{article}
\usepackage[utf8]{inputenc}
\usepackage[spanish]{babel}
\usepackage{graphicx}
\usepackage{hyperref}
\usepackage{geometry}
\usepackage{fancyhdr}
\usepackage{xcolor}
\usepackage{listings}
\usepackage{tcolorbox}
\usepackage{enumitem}

\geometry{margin=2.5cm}

\definecolor{primaryblue}{HTML}{31436B}
\definecolor{codebg}{HTML}{1E1E1E}
\definecolor{codetext}{HTML}{D4D4D4}
\definecolor{keyword}{HTML}{569CD6}
\definecolor{string}{HTML}{CE9178}
\definecolor{comment}{HTML}{6A9955}

\lstset{
    backgroundcolor=\color{codebg},
    basicstyle=\ttfamily\small\color{codetext},
    keywordstyle=\color{keyword},
    stringstyle=\color{string},
    commentstyle=\color{comment},
    breaklines=true,
    frame=single,
    rulecolor=\color{gray},
    showstringspaces=false,
    tabsize=4
}

\hypersetup{
    colorlinks=true,
    linkcolor=primaryblue,
    urlcolor=primaryblue
}

\pagestyle{fancy}
\fancyhf{}
\fancyhead[L]{\textbf{Manual de Despliegue - SPGCU}}
\fancyhead[R]{\thepage}
\fancyfoot[C]{Universidad Nacional de Piura}

\title{
    \vspace{-2cm}
    % \includegraphics[width=4cm]{logo_unp.png} \\ % Descomentar cuando tengas el logo
    \vspace{1cm}
    \vspace{1cm}
    \textbf{\Huge Manual de Despliegue} \\
    \vspace{0.5cm}
    \textbf{\Large Sistema de Programación y Gestión del Comedor Universitario} \\
    \vspace{0.3cm}
    \textbf{(SPGCU)}
}
\author{
    \textbf{Área de Sistemas} \\
    Universidad Nacional de Piura
}
\date{Diciembre 2025}

\begin{document}

\maketitle
\thispagestyle{empty}
\newpage

\tableofcontents
\newpage

% ============================================================
\section{Introducción}
% ============================================================

Este manual describe los procedimientos para desplegar el Sistema SPGCU en diferentes entornos: desarrollo local y servidor de producción.

\subsection{Stack Tecnológico}
\begin{itemize}
    \item \textbf{Backend}: Laravel 10 (PHP 8.1+)
    \item \textbf{Frontend}: React 18 + Inertia.js
    \item \textbf{Base de Datos}: MySQL 8.0 / MariaDB 10.6+
    \item \textbf{Build Tool}: Vite
    \item \textbf{Servidor Web}: Nginx (producción) / PHP Built-in (desarrollo)
\end{itemize}

\subsection{Repositorio del Proyecto}
\begin{lstlisting}[language=bash]
git clone https://github.com/universidad/spgcu_system.git
cd spgcu_system
\end{lstlisting}

% ============================================================
\section{Requisitos del Sistema}
% ============================================================

\subsection{Para Desarrollo Local (Windows/Mac/Linux)}
\begin{itemize}
    \item PHP 8.1 o superior
    \item Composer 2.x
    \item Node.js 18+ y NPM 9+
    \item MySQL 8.0 o MariaDB 10.6+
    \item Git
\end{itemize}

\subsection{Para Servidor de Producción (Ubuntu 22.04)}
\begin{itemize}
    \item Ubuntu Server 22.04 LTS
    \item PHP 8.1-FPM con extensiones: mbstring, xml, bcmath, curl, zip, gd, mysql
    \item Nginx 1.18+
    \item MySQL 8.0 o MariaDB 10.6+
    \item Certbot (para SSL)
    \item Supervisor (para colas)
\end{itemize}

% ============================================================
\section{Despliegue en Desarrollo Local}
% ============================================================

\subsection{Paso 1: Clonar el Repositorio}
\begin{lstlisting}[language=bash]
git clone https://github.com/universidad/spgcu_system.git
cd spgcu_system
\end{lstlisting}

\subsection{Paso 2: Instalar Dependencias PHP}
\begin{lstlisting}[language=bash]
composer install
\end{lstlisting}

\subsection{Paso 3: Instalar Dependencias JavaScript}
\begin{lstlisting}[language=bash]
npm install
\end{lstlisting}

\subsection{Paso 4: Configurar Variables de Entorno}
\begin{lstlisting}[language=bash]
cp .env.example .env
php artisan key:generate
\end{lstlisting}

Editar el archivo \texttt{.env} con los siguientes valores:

\begin{lstlisting}
APP_NAME="SPGCU"
APP_ENV=local
APP_DEBUG=true
APP_URL=http://localhost:8000

DB_CONNECTION=mysql
DB_HOST=127.0.0.1
DB_PORT=3306
DB_DATABASE=spgcu_db
DB_USERNAME=root
DB_PASSWORD=tu_password
\end{lstlisting}

\subsection{Paso 5: Crear Base de Datos}
\begin{lstlisting}[language=sql]
CREATE DATABASE spgcu_db CHARACTER SET utf8mb4 COLLATE utf8mb4_unicode_ci;
\end{lstlisting}

\subsection{Paso 6: Ejecutar Migraciones y Seeders}
\begin{lstlisting}[language=bash]
php artisan migrate --seed
\end{lstlisting}

\subsection{Paso 7: Iniciar Servidores de Desarrollo}

\textbf{Terminal 1 - Backend Laravel:}
\begin{lstlisting}[language=bash]
php artisan serve
\end{lstlisting}

\textbf{Terminal 2 - Frontend Vite:}
\begin{lstlisting}[language=bash]
npm run dev
\end{lstlisting}

\subsection{Paso 8: Acceder al Sistema}
Abrir navegador en: \texttt{http://localhost:8000}

\begin{tcolorbox}[colback=blue!5,colframe=blue!50,title=Credenciales por Defecto]
\textbf{Administrador:} \\
DNI: 12345678 \\
Contraseña: password123
\end{tcolorbox}

% ============================================================
\section{Despliegue en Producción (Ubuntu + Nginx)}
% ============================================================

\subsection{Paso 1: Preparar el Servidor}

\subsubsection{Actualizar Sistema}
\begin{lstlisting}[language=bash]
sudo apt update && sudo apt upgrade -y
\end{lstlisting}

\subsubsection{Instalar PHP y Extensiones}
\begin{lstlisting}[language=bash]
sudo apt install -y php8.1-fpm php8.1-cli php8.1-mysql \
    php8.1-mbstring php8.1-xml php8.1-bcmath php8.1-curl \
    php8.1-zip php8.1-gd php8.1-intl
\end{lstlisting}

\subsubsection{Instalar Nginx}
\begin{lstlisting}[language=bash]
sudo apt install -y nginx
\end{lstlisting}

\subsubsection{Instalar MySQL}
\begin{lstlisting}[language=bash]
sudo apt install -y mysql-server
sudo mysql_secure_installation
\end{lstlisting}

\subsubsection{Instalar Composer}
\begin{lstlisting}[language=bash]
curl -sS https://getcomposer.org/installer | php
sudo mv composer.phar /usr/local/bin/composer
\end{lstlisting}

\subsubsection{Instalar Node.js}
\begin{lstlisting}[language=bash]
curl -fsSL https://deb.nodesource.com/setup_18.x | sudo -E bash -
sudo apt install -y nodejs
\end{lstlisting}

\subsection{Paso 2: Clonar y Configurar el Proyecto}

\begin{lstlisting}[language=bash]
cd /var/www
sudo git clone https://github.com/universidad/spgcu_system.git
sudo chown -R www-data:www-data spgcu_system
cd spgcu_system
\end{lstlisting}

\subsection{Paso 3: Instalar Dependencias}
\begin{lstlisting}[language=bash]
composer install --optimize-autoloader --no-dev
npm install
npm run build
\end{lstlisting}

\subsection{Paso 4: Configurar Variables de Entorno}
\begin{lstlisting}[language=bash]
cp .env.example .env
php artisan key:generate
nano .env
\end{lstlisting}

Configurar para producción:
\begin{lstlisting}
APP_NAME="SPGCU"
APP_ENV=production
APP_DEBUG=false
APP_URL=https://spgcu.unp.edu.pe

DB_CONNECTION=mysql
DB_HOST=127.0.0.1
DB_PORT=3306
DB_DATABASE=spgcu_production
DB_USERNAME=spgcu_user
DB_PASSWORD=ContraseñaSegura123!

QUEUE_CONNECTION=database
SESSION_DRIVER=database
\end{lstlisting}

\subsection{Paso 5: Configurar Base de Datos}
\begin{lstlisting}[language=bash]
sudo mysql -u root -p
\end{lstlisting}

\begin{lstlisting}[language=sql]
CREATE DATABASE spgcu_production 
    CHARACTER SET utf8mb4 
    COLLATE utf8mb4_unicode_ci;

CREATE USER 'spgcu_user'@'localhost' 
    IDENTIFIED BY 'ContraseñaSegura123!';

GRANT ALL PRIVILEGES ON spgcu_production.* 
    TO 'spgcu_user'@'localhost';

FLUSH PRIVILEGES;
EXIT;
\end{lstlisting}

\subsection{Paso 6: Ejecutar Migraciones}
\begin{lstlisting}[language=bash]
php artisan migrate --seed --force
\end{lstlisting}

\subsection{Paso 7: Optimizar Laravel}
\begin{lstlisting}[language=bash]
php artisan config:cache
php artisan route:cache
php artisan view:cache
php artisan storage:link
\end{lstlisting}

\subsection{Paso 8: Configurar Permisos}
\begin{lstlisting}[language=bash]
sudo chown -R www-data:www-data /var/www/spgcu_system
sudo chmod -R 755 /var/www/spgcu_system
sudo chmod -R 775 /var/www/spgcu_system/storage
sudo chmod -R 775 /var/www/spgcu_system/bootstrap/cache
\end{lstlisting}

\subsection{Paso 9: Configurar Nginx}

Crear archivo de configuración:
\begin{lstlisting}[language=bash]
sudo nano /etc/nginx/sites-available/spgcu
\end{lstlisting}

Contenido:
\begin{lstlisting}
server {
    listen 80;
    server_name spgcu.unp.edu.pe;
    root /var/www/spgcu_system/public;

    add_header X-Frame-Options "SAMEORIGIN";
    add_header X-Content-Type-Options "nosniff";

    index index.php;

    charset utf-8;

    location / {
        try_files $uri $uri/ /index.php?$query_string;
    }

    location = /favicon.ico { access_log off; log_not_found off; }
    location = /robots.txt  { access_log off; log_not_found off; }

    error_page 404 /index.php;

    location ~ \.php$ {
        fastcgi_pass unix:/var/run/php/php8.1-fpm.sock;
        fastcgi_param SCRIPT_FILENAME $realpath_root$fastcgi_script_name;
        include fastcgi_params;
    }

    location ~ /\.(?!well-known).* {
        deny all;
    }
}
\end{lstlisting}

Activar configuración:
\begin{lstlisting}[language=bash]
sudo ln -s /etc/nginx/sites-available/spgcu /etc/nginx/sites-enabled/
sudo nginx -t
sudo systemctl restart nginx
\end{lstlisting}

\subsection{Paso 10: Configurar SSL con Certbot}
\begin{lstlisting}[language=bash]
sudo apt install -y certbot python3-certbot-nginx
sudo certbot --nginx -d spgcu.unp.edu.pe
\end{lstlisting}

\subsection{Paso 11: Configurar Tareas Programadas (Cron)}
\begin{lstlisting}[language=bash]
sudo crontab -e
\end{lstlisting}

Agregar:
\begin{lstlisting}
* * * * * cd /var/www/spgcu_system && php artisan schedule:run >> /dev/null 2>&1
\end{lstlisting}

\subsection{Paso 12: Configurar Supervisor para Colas}

Crear configuración:
\begin{lstlisting}[language=bash]
sudo apt install -y supervisor
sudo nano /etc/supervisor/conf.d/spgcu-worker.conf
\end{lstlisting}

Contenido:
\begin{lstlisting}
[program:spgcu-worker]
process_name=%(program_name)s_%(process_num)02d
command=php /var/www/spgcu_system/artisan queue:work --sleep=3 --tries=3
autostart=true
autorestart=true
user=www-data
numprocs=2
redirect_stderr=true
stdout_logfile=/var/www/spgcu_system/storage/logs/worker.log
\end{lstlisting}

Activar:
\begin{lstlisting}[language=bash]
sudo supervisorctl reread
sudo supervisorctl update
sudo supervisorctl start spgcu-worker:*
\end{lstlisting}

% ============================================================
\section{Comandos de Mantenimiento}
% ============================================================

\subsection{Limpiar Caché}
\begin{lstlisting}[language=bash]
php artisan cache:clear
php artisan config:clear
php artisan route:clear
php artisan view:clear
\end{lstlisting}

\subsection{Actualizar Aplicación}
\begin{lstlisting}[language=bash]
cd /var/www/spgcu_system
git pull origin main
composer install --optimize-autoloader --no-dev
npm install && npm run build
php artisan migrate --force
php artisan config:cache
php artisan route:cache
php artisan view:cache
sudo supervisorctl restart spgcu-worker:*
\end{lstlisting}

\subsection{Procesar Inasistencias Manualmente}
\begin{lstlisting}[language=bash]
# Procesar el dia anterior
php artisan comedor:marcar-inasistencias

# Procesar menus terminados hoy (tiempo real)
php artisan comedor:marcar-inasistencias --realtime
\end{lstlisting}

\subsection{Ver Logs}
\begin{lstlisting}[language=bash]
tail -f /var/www/spgcu_system/storage/logs/laravel.log
\end{lstlisting}

\subsection{Backup de Base de Datos}
\begin{lstlisting}[language=bash]
mysqldump -u spgcu_user -p spgcu_production > backup_$(date +%Y%m%d).sql
\end{lstlisting}

% ============================================================
\section{Resolución de Problemas Comunes}
% ============================================================

\subsection{Error 500 - Internal Server Error}
\begin{enumerate}
    \item Verificar permisos de \texttt{storage/} y \texttt{bootstrap/cache/}.
    \item Revisar logs: \texttt{tail -f storage/logs/laravel.log}.
    \item Limpiar caché: \texttt{php artisan optimize:clear}.
\end{enumerate}

\subsection{Página en Blanco}
\begin{enumerate}
    \item Verificar que \texttt{APP\_DEBUG=true} temporalmente.
    \item Verificar configuración de PHP-FPM.
    \item Revisar logs de Nginx: \texttt{/var/log/nginx/error.log}.
\end{enumerate}

\subsection{Error de Conexión a Base de Datos}
\begin{enumerate}
    \item Verificar credenciales en \texttt{.env}.
    \item Verificar que MySQL esté corriendo: \texttt{systemctl status mysql}.
    \item Probar conexión: \texttt{mysql -u spgcu\_user -p}.
\end{enumerate}

\subsection{Assets No Cargan (CSS/JS)}
\begin{enumerate}
    \item Ejecutar \texttt{npm run build}.
    \item Verificar \texttt{APP\_URL} en \texttt{.env}.
    \item Limpiar caché del navegador.
\end{enumerate}

\subsection{Escáner QR No Funciona}
\begin{enumerate}
    \item El escáner requiere \textbf{HTTPS} en producción.
    \item Verificar permisos de cámara en el navegador.
    \item Probar en \texttt{localhost} primero.
\end{enumerate}

% ============================================================
\section{Diagrama de Arquitectura}
% ============================================================

\begin{figure}[h]
    \centering
    % \includegraphics[width=0.9\textwidth]{arquitectura.png}
    \caption{Arquitectura del Sistema SPGCU}
    \label{fig:arquitectura}
\end{figure}

\begin{verbatim}
+------------------+     +------------------+     +------------------+
|   Cliente Web    | --> |      Nginx       | --> |   Laravel App    |
| (Navegador/Móvil)|     | (Reverse Proxy)  |     | (PHP-FPM 8.1)    |
+------------------+     +------------------+     +------------------+
                                                          |
                         +------------------+             |
                         |     MySQL 8.0    | <-----------+
                         | (Base de Datos)  |
                         +------------------+
\end{verbatim}

% ============================================================
\section{Contacto y Soporte}
% ============================================================

\begin{itemize}
    \item \textbf{Área de Sistemas - UNP}
    \item Correo: sistemas@unp.edu.pe
    \item Teléfono: (073) 123456 Anexo 200
\end{itemize}

\vspace{2cm}
\hrule
\vspace{0.5cm}
\centering
\textit{Documento generado el \today}

\end{document}
